\documentclass[pdftex,11pt,a4paper]{book}

\usepackage[spanish]{babel}
\usepackage[utf8]{inputenc}
\usepackage[pdftex]{graphicx}
\usepackage{url}
\usepackage[top=1.5in, bottom=1in, left=1in, right=1in]{geometry}
\usepackage{hyperref}
\usepackage{cite}
\usepackage{float}
\usepackage{eurosym} % para el euro
\usepackage{enumerate}
\usepackage{listings}    

\newcommand{\HRule}{\rule{\linewidth}{0.5mm}}

\renewcommand\figurename{Imagen}


\begin{document}

\begin{titlepage}
\begin{center}

% Upper part of the page. The '~' is needed because \\
% only works if a paragraph has started.
% -----
%\includegraphics[width=0.35\textwidth]{img/marca-uji-2955.png}~\\[2cm]
\includegraphics{img/marca-uji-2955.png}~\\[3.0cm]

\textsc{\LARGE Grado en Ingeniería Informática}\\[1.5cm]

\textsc{\LARGE Diseño e Implementación de Sistemas de Información}\\[1.5cm]
%%%%%%\textsc{\LARGE Versión Provisional}\\[1.5cm]

% Title
\HRule \\[0.4cm]
{ \huge \bfseries Sistema de soporte a la gestión de la Asignación de Proyectos en Empresas \\[0.4cm] }

\HRule \\[1.5cm]

% Author and supervisor
\begin{minipage}{0.4\textwidth}
\begin{flushleft} \large
\emph{Autor:}\\
Víctor \textsc{Sánchez Terrasa}
\end{flushleft}
\end{minipage}
\begin{minipage}{0.4\textwidth}
\begin{flushright} \large
\emph{Tutor académico:} \\
María de los Ángeles \textsc{López Malo}
\end{flushright}
\end{minipage}

\vfill

% Bottom of the page
% -----
% Sustituir los signos "\_" por lo que corresponda. 
{\large Fecha de lectura: 16 de julio de 2018\\
Curso académico 2018/2019}

\end{center}
\end{titlepage}
\setlength{\parskip}{\baselineskip}



% ------------------- Página resumen ---------------------

\thispagestyle{empty} % página sin numerar

\clearpage % Resumen y Palabras clave en página 2.

\section*{Resumen}

En el grado de Ingeniería Informática se necesita un sistema que pueda dar soporte a la gestión de las estancias en prácticas, en estas estancias se incluye también el TFG y la defensa el mismo ante el tribunal, aunque en este caso el alcance del proyecto sólo llegará a la gestionar la asignación de los estudiantes a una estancia en prácticas. 

\thispagestyle{empty} % página sin numerar

\cleardoublepage


% ------------------- Página índice ---------------------


\pagestyle{plain} % todas las páginas numeradas, sin cabeceras. Sustituir por \pagestyle{headings} para añadir cabeceras a las páginas. 

\tableofcontents

\cleardoublepage

% ------------------- Cuerpo de la memoria ---------------------

\chapter{Introducción}

\section{Contexto del proyecto}

En el grado de Ingeniería Informática se necesita un sistema que pueda dar soporte a la gestión de las estancias en prácticas, en estas estancias se incluye también el TFG y la defensa el mismo ante el tribunal, aunque en este caso el alcance del proyecto sólo llegará a la gestionar la asignación de los estudiantes a una estancia en prácticas.

\section{Motivación del proyecto}

El objetivo principal, y estratégico, de este producto es proporcionar apoyo a la gestión propia de la asignatura EI1054: Estancias y Trabajo Final de Grado (TFG) de Ingeniería en Informática.

A nivel táctico y operativo el producto permite agilizar y mejorar la gestión propia de la asignatura EI1054 (Estancias en prácticas y trabajo final de grado). Proporciona un entorno accesible a las empresas y los supervisores para que la gestión de los TFG se pueda llevar a cabo de manera más colaborativa. Facilita el trabajo de revisión y de aceptación de propuestas de TFG. Mejora la publicación y consulta de las ofertas para la realización de los TFG del grado. Finalmente, proporciona transparencia en el proceso de las asignaciones de estancias y TFG.

SAPE se usa únicamente para gestionar las estancias y el TFG del grado de Ingeniería Informática una vez las empresas han creado una oferta para el grado en el sistema central de gestión de la UJI: el IGLU. SAPE se conecta con IGLU para evitar duplicidades de tareas y facilitar el trabajo de la CEiTFG y los CCG. 

En SAPE el usuario principal, y propietarios del producto, es la CEiTFG. Además, son usuarios directos a los alumnos matriculados en la asignatura EI1054, los CCG y las personas de contacto de las empresas. Finalmente, los profesores tutores y los supervisores de las estancias en las empresas son usuarios que sólo reciben información.

SAPE incluye la siguiente funcionalidad:
\begin{itemize}
\item Cuando una empresa introduzca una oferta de estancias para EI en IGLU, deberá acceder a SAPE y completar los datos del TFG. La conexión será automática con el mismo usuario y pwd que tienen dado de alta en el IGLU.
\item La CEiTFG y los CCG puede aprobar propuestas y solicitar cambios o mejoras. SAPE generar una orden para el envío de mails a los contactos de las empresas. Los mails indica qué información se ha pedido y se registrará los datos del envío para que quede constancia.
\item La CEiTFG y los CCG podrán modificar y rechazar propuestas, registrando el motivo. SAPE generará una orden para el envío de un mail que informe la persona de contacto de la empresa del hecho y del motivo, y registrará información del envío para qué quede constancia.
\item Las personas de contacto de las empresas podrán modificar sus propuestas, mientras no estén aceptadas, usando el mismo usuario y pwd que tienen dado de alta en IGLU.
\item En una fecha establecida, la CEiTFG podrá hacer públicas las propuestas revisadas y aprobadas que hage en SAPE hasta esa fecha. Se podrán añadir propuestas aprobadas con posterioridad.
\item Los alumnos matriculados en la asignatura podrán consultar las propuestas de proyectos y la información de interés sobre las empresas.
\item Los alumnos podrán indicar sus preferencias de manera ordenada incluyendo comentarios.
\item La CEiTFG podrá asignar una propuesta de proyecto y un tutor a un alumno. Para poder hacer esta asignación podrán consultar cuál es la ordenación de los alumnos segundo su expediente académico en IGLU.
\item Los alumnos podrán consultar la propuesta de asignación provisional y tendrán un plazo para pedir cambios. Estas solicitudes de cambio quedarán registradas junto con la propuesta inicial.
\item La CEiTFG podrá revisar los cambios solicitados, hacer cambios y volver a publicar la propuesta.
\item La CEiTFG podrá anular una propuesta de asignación para solucionar incidencias puntuales.
\item La CEiTFG podrá activar el traspaso al IGLU de las asignaciones aceptadas.

\end{itemize}

Para coordinar el trabajo realizado por todos los miembros del equipo, lo que hemos hecho son quedadas grupales en las que hemos realizado diferentes partes del proyecto de forma conjunta a mes de distribuir algunas tareas individuales como puede ser la creación de la base de datos a partir del diseño físico o la realización de partes del código como el validadores. Además de la utilización del drive para poder poner en común los cambios realizados de forma individual en casa.

%\section{Estructura de la memoria}

%En este documento se encuentran las etapas del desarrollo de la aplicación pudiendo encontrar en el capítulo 2 la definición del proyecto, como también los requisitos y una descripción de las principales tecnologías usadas en el proyecto, junto con su motivo de uso. El capítulo 3 muestra la metodología que la empresa emplea, la misma empleada en el proyecto, así como también la planificación y el seguimiento que se ha realizado del proyecto. En el capítulo 4 ya se encuentra más información de campo de la informática como el análisis realizado por la aplicación, además del diseño realizado posteriormente del análisis, tanto de la arquitectura del sistema, como de la interfaz. El documento continúa con el capítulo 5 en el que se muestra la implementación de la aplicación, como la puesta en marcha y los errores encontrados en ésta. Finalmente, en el capítulo 6, se tiene a disposición las conclusiones del producto.

\chapter{Diseño de la base de datos}

\section{Modelo UML proporcionado en clase}

Para un correcto desarrollo de la base de datos para el proyecto es necesario que tengamos un modelo UML en el que apoyarnos para un correcto desarrollo. Por esta razón hicimos uso del modelo UML que nos dieron a clase el cual es el que se muestra en la imagen siguiente.

\begin{figure}[h]
\begin{center}
\includegraphics[width=\textwidth]{img/uml_enunciado.jpg}
\caption{\label{uml_propio} UML creado propuesto en el enunciado.}
\end{center}
\end{figure}

\section{Modelo UML definido para nuestro proyecto}
Una vez tenemos el modelo UML de nuestro proyecto, el siguiente paso fue la
realización de un diseño lógico en el que se ven todas las relaciones entre las diferentes clases que necesitamos para la implementación además de los diferentes elementos que contienen cada una. Este diseño lógico de por ver en la siguiente imagen.

\begin{figure}[h]
\begin{center}
\includegraphics[width=\textwidth]{img/uml_propio}
\caption{\label{uml_propio} UML creado a partir del expuesto en el enunciado.}
\end{center}
\end{figure}

Una vez tenemos el modelo UML de nuestro proyecto, el siguiente paso fue la
realización de un diseño lógico en el que se ven todas las relaciones entre las diferentes clases que necesitamos para la implementación además de los diferentes elementos que contienen cada una. Este diseño lógico de por ver en la siguiente imagen.

\section{Diseño físico}

\lstinputlisting[language=SQL]{disenyo_trabajo.sql}



\chapter{Diseño de interfaces de usuario.}

Para el diseño de interfaces de usuario hemos hecho un proceso de diseño centrado en el usuario en el que hemos llevado a cabo los siguientes pasos:
\begin{itemize}
\item \textbf{Planificación:} hemos identificado los objetivos del sitio web, las necesidades, requerimientos y objetivos de la usuario. Además hemos definido definen los objetivos técnicos, equipo de trabajo y presupuesto.

\item \textbf{Diseño conceptual:} hicimos el sitemas del proyecto desarrollado en el que se ven las diferentes opciones a la hora de navegar por la página web desarrollada según el tipo de ro, que tengas dentro del proyecto ya sea empresa, DCC, BTC, o un alumno.

\item \textbf{Diseño visual y definición de estilos:} acordamos una serie de características visuales que se mantienen durante todas las páginas por las que se pueden navegar a amas de otras características individuales, que se pueden ver en las siguientes imágenes de interfaces que hemos diseñado.

\end{itemize}

\begin{figure}[h]
\begin{center}
\includegraphics[width=\textwidth]{img/arbol_web}
\caption{\label{mapa_web} Mapa web con el acceso a los distintos contenidos.}
\end{center}
\end{figure}


A continuación se puede observar tres ejemplos de interfaces que hemos desarrollado para nuestra página web, y que más adelante hemos evaluado.


A continuación se presenta el informe del análisis heurístico realizado sobre el sitio SAPE con fecha 25/06/2018


\chapter{Implementación}
\section{Decisiones de implementación}

Durante el diseño de la página web realizada en la asignatura EI1027 - Diseño y implementación de sistemas de información, hemos realizado una estructura MVC (Modelo Vista Controlador) en el cual hemos desarrollado todas las clases necesarias para el proyecto en las cuales se instancian todos los posibles usuarios de la pagina web, también hay que mencionar que hemos añadido diferentes roles en la base de datos, ya que dependiendo del rol que se tenga en la página podrás acceder a cierta información. Por otra parte también hemos realizado varia páginas html, las cuales son las distintas páginas por las que los usuarios podrán navegar dependiendo de qué rol tienen. Por último se ha desarrollado una serie de controladores, los cuales, son los encargados de controlar el acceso a la base de datos.
\section{Control de errores}

Una vez se ha implementado el proyecto ha sido necesario un control de errores, ya que los usuarios no son perfectos y por lo tanto pueden tener errores a la hora de introducir datos en la página, para ello hemos implementado una serie de validadores para asegurarnos que los datos introducidos por el usuario son válidos y que, en caso de producirse un error durante su introducción, se informe de forma adecuada al usuario para que lo rectifique. Por último también ha sido necesario insertar ciertas restricciones para la correcta actualización de la base de datos. 

\section{Listado de paquetes y clases}

En la figura 7 se muestra la estructura del proyecto, el paquete raíz está formado por tres paquetes: controller, dao y modelo.
En el paquete controller es donde se almacenan los controladores de todas las clases y los validadores que se han implementado para la aplicación.
En el paquete dao están las clases que acceden a nuestra base de datos.
En el paquete modelo se encuentran las clases que hemos utilizado en nuestro diagrama de clases para la implementación.

Para acabar comentar que todas las clases que aparecen en los paquetes mencionados anteriormente también aparecen el diagrama de clases elaborado al principio del desarrollo del proyecto.



\chapter{Conclusiones} 
Como conclusión hay que decir que el proyecto desarrollado en la asignatura EI1027 - Diseño e Implementación de Sistemas de información ha sido un proyecto que se ha desarrollado durado meses y que ha tenido una carga de trabajo bastante elevada, pero por otro lado es un proyecto muy completo en el que hemos visto todas las etapas para la realización de un sitio web de forma correcta, además de que ha sido un proyecto en el que hemos aprendido muchas herramientas, las cuales, seguro que utilizaremos en un futuro .









%Se pueden presentar conclusiones en varios aspectos: 
%en el ámbito formativo (sobre lo que has aprendido),
%en el ámbito profesional (sobre la experiencia en la empresa)
%y en el ámbito personal (sobre tu experiencia personal).
%En las conclusiones, además de las consideraciones personales, académicas o profesionales que el alumno quiera comentar, 
%se pueden incluir posibles extensiones del proyecto, así como la viabilidad comercial o empresarial cuando proceda.

Durante el desarrollo de la aplicación, se han adquirido nuevas competencias que se desconocían, como la resolución rápida de conflictos, la capacidad de documentar aplicaciones y las sinergias de trabajar con otros departamentos.

Además la aplicación ha conseguido evolucionar más sin la necesidad de emplear mucho tiempo en su desarrollo, añadiendo funcionalidades como puede ser realizar un volcado de datos a un fichero, tener un control de registros de eventos de las solicitudes recibidas y almacenarlas en un fichero intermediario entre la aplicación y el \emph{PLC}. No solo se han cumplido los objetivos básicos inicialmente establecidos, sino que además se han incluido otros que pueden ser de gran ayuda.

Sin embargo, en una futura revisión, se deberían corregir algunas carencias del sistema:
\begin{itemize}
\item Simulación del \emph{PLC} más real.
\item Añadir control de versión del \emph{PLC}.
\item Permitir almacenar los datos en la ubicación que el usuario desee.
\item Permitir leer los datos en la ubicación que el usuario desee.
\item Permitir cambiar el puerto de escucha por defecto.
\item Permitir almacenar una copia del registro de eventos.
\item Autoaprendizaje mediante inteligencia artificial para simular con la mayor precisión posible.
\end{itemize}


% ------------------- Bibliografia ---------------------
\addcontentsline{toc}{chapter}{Bibliografía}
%\bibliographystyle{plane}
%\bibliography{MemoriaTecnicaBibliografia.bib}

\begin{thebibliography}{X}
\bibitem{CasPractic}
 \textsc{EI1023 - Fonaments d’enginyeria del programari},
 \textit{Cas per al treball de pràctiques de l'assignatura},
 \url{https://docs.google.com/document/d/16aIXg8UdvHcTzhVyMVpaHZVP_KefRawoehEIr81dRNc/},
  29 de octubre de 2018.
\bibitem{presentacion}
 \textsc{EI1027 - Disseny i Implementació de Sistemes d'informació},
 \textit{Introducció a la usabilitat i el disseny d'interfícies},
 \url{https://docs.google.com/presentation/d/18kpzN31lZoxK2hXcdZPjD_FLBBdm8KG5-db__kYjlKk/},
  29 de octubre de 2018.
\bibitem{ejercicio}
 \textsc{EI1027 - Disseny i Implementació de Sistemes d'informació},
 \textit{Exercici Disseny d’interfícies d’usuari centrat en l’usuari},
 \url{https://docs.google.com/document/d/1g6bx4PCpFXTPRI6hy8UcK9XvFo6obJrpJBEnpC1UlSk/},
  29 de octubre de 2018.
\bibitem{evaluacion}
 \textsc{EI1027 - Disseny i Implementació de Sistemes d'informació},
 \textit{Avaluació de llocs web},
 \url{https://docs.google.com/presentation/d/1wMODQ3NZlBf_bLPlPbN0X8dh-wjywwT8cy9EgVYpTh4/},
  29 de octubre de 2018.
\end{thebibliography}


% ------------------- Anexos ---------------------

\appendix
%\renewcommand\appendixname{Anexo}

% ---- Primer Anexo ----
%\chapter{Estudio detallado de...}

%\section{Definición}

%\section{Aplicaciones}

% ---- Segundo Anexo ----
%\chapter{Tablas de ...}

\end{document}
